% !TEX root = sidd-cv.tex

\section{Teaching}
\noindent
\textbf{CSE 478 Autonomous Robotics}\hfill Winter 2017-\\
\textbf{Paul G. Allen School for Computer Science \& Engineering}\\
\parbox[t]{0.75\textwidth}{
Brand new undergraduate-level robotics course on robotics in the real world. The course covers state estimation (particle filters, motion models, sensor models etc), planning/control (search based planners, lattice based planners, trajectory following techniques etc), and perception and learning (object detection, learning from demonstrations etc.). Student teams implement algorithms on the RACECAR platform developed by Prof. Srinivasa for the course.
}\\
\\
\noindent
\textbf{CSE 599 Advanced Robotics}\hfill Fall 2017\\
\textbf{Paul G. Allen School for Computer Science \& Engineering}\\
\parbox[t]{0.75\textwidth}{
Brand new graduate-level robotics course on motion planning algorithms. The course covers the Piano Movers Problem, sampling-based planning, minimum dispersion graphs, efficient search, lazy and anytime planning, planning under uncertainty with application to mobile manipulators and humanoid robots, with a focus on algorithmic foundations and theorem proving.
}\\
\\
\noindent
\textbf{16-843 Manipulation Algorithms}\hfill Fall 2012-16\\
\textbf{The Robotics Institute, Carnegie Mellon University}\\
\parbox[t]{0.75\textwidth}{
Brand new graduate-level robotics course on the theory and algorithms that enable robots to physically manipulate their world. The course covers the geometry of manipulation configuration spaces, motion planning in these spaces, synthesizing robust and stable grasps for dexterous hands, reconfiguring clutter, task-level planning of multi-stage manipulation, physics-based actions, and addressing perception and model uncertainty, with application to mobile manipulators and humanoid robots.
}\\
\\
\noindent
\textbf{16-662 Robot Autonomy}\hfill Spring 2012-16\\
\textbf{The Robotics Institute, Carnegie Mellon University}\\
\parbox[t]{0.75\textwidth}{
Brand new graduate-level robotics course on manipulation, motion planning, perception, navigation,
and machine learning algorithms for mobile manipulators. The course covers theory and algorithms,
and has a strong hands-on component where students implement their assignments and class projects on a real mobile manipulation platform.
}\\
\\
\textbf{16-741 Mechanics of Manipulation}\hfill Spring 2009\\
\textbf{The Robotics Institute, Carnegie Mellon University}\\
Co-taught with Matt Mason\\
\parbox[t]{0.75\textwidth}{
Graduate-level robotics core course on model-based robotic manipulation.
To develop techniques for rigid body mechanics, kinematic constraint, Coulomb
friction, gravity, and impact, and apply these techniques to manipulation
problems including picking and placing, parts orienting, assembly, and mobile
manipulation. 
}\\
