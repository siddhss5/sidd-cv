% !TEX root = sidd-cv.tex

\section{Teaching}
\noindent
\textbf{16-843 Manipulation Algorithms}\hfill Fall 2012-\\
\textbf{The Robotics Institute, Carnegie Mellon University}\\
\parbox[t]{0.75\textwidth}{
Graduate-level robotics course on the theory and algorithms that enable robots to physically manipulate their world. The course covers the geometry of manipulation configuration spaces, motion planning in these spaces, synthesizing robust and stable grasps for dexterous hands, reconfiguring clutter, task-level planning of multi-stage manipulation, physics-based actions, and addressing perception and model uncertainty, with application to mobile manipulators and humanoid robots.
}\\
\\
\noindent
\textbf{16-662 Robot Autonomy}\hfill Spring 2012-\\
\textbf{The Robotics Institute, Carnegie Mellon University}\\
\parbox[t]{0.75\textwidth}{
Graduate-level robotics course on manipulation, motion planning, perception, navigation,
and machine learning algorithms for mobile manipulators. The course covers theory and algorithms,
and has a strong hands-on component where students implement their assignments and class projects on a real mobile manipulation platform.
}\\
\\
\textbf{16-741 Mechanics of Manipulation}\hfill Spring 2009\\
\textbf{The Robotics Institute, Carnegie Mellon University}\\
Co-taught with Matt Mason\\
\parbox[t]{0.75\textwidth}{
Graduate-level robotics core course on model-based robotic manipulation.
To develop techniques for rigid body mechanics, kinematic constraint, Coulomb
friction, gravity, and impact, and apply these techniques to manipulation
problems including picking and placing, parts orienting, assembly, and mobile
manipulation. 
}\\
