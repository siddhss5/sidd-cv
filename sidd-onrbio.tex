\documentclass[10pt]{article}

\RequirePackage{color,graphicx}
\usepackage{supertabular} 				%for Grades
\usepackage{titlesec}					%custom \section
\usepackage{palatino}
% Below sets font to Arial; acceptable fonts include 10pt Arial (space saver) or 11pt Times New Roman or Computer Modern 
%\renewcommand{\rmdefault}{phv} 
%\renewcommand{\sfdefault}{phv} 

\usepackage{bibentry}
\bibliographystyle{plain} 
\nobibliography*


\usepackage{tabularx}
%Setup hyperref package, and colours for links
\usepackage{hyperref}
\hypersetup{colorlinks,breaklinks,urlcolor=blue, linkcolor=blue}
\usepackage{url}
\urlstyle{sf}

%CV Sections inspired by: 
%http://stefano.italians.nl/archives/26
\titleformat{\section}{\bf\large\raggedright}{}{0em}{}[\titlerule]
\titlespacing{\section}{0pt}{5pt}{5pt}
\titleformat{\subsection}{\bf\raggedright}{}{0em}{}
\titlespacing{\subsection}{0pt}{0pt}{10pt}
% Tweak page extents
\addtolength{\topmargin}{-8em}
\addtolength{\textheight}{8em}
\addtolength{\oddsidemargin}{-8em}
\addtolength{\evensidemargin}{-8em}
\addtolength{\textwidth}{16em}

% Using svn:keyword
% svn propset svn:keywords 'Date' cv-sidd-10.tex
\newcommand{\svn}[1]{\svnsub#1}
\def\svnsub$#1${#1}

%--------------------BEGIN DOCUMENT----------------------
\begin{document}

\pagestyle{empty} % non-numbered pages

%--------------------TITLE-------------
\par{\centering
		{\bf\LARGE Prof. Siddhartha Srinivasa
	}\bigskip\par}


%--------------------LAST MODIFIED-------------
%\let\thefootnote\relax\footnotetext{Last modified: \svn{$Date: 2012-12-09 21:46:45 -0500 (Sun, 09 Dec 2012) $}} 

%--------------------SECTIONS-----------------------------------
\begin{center}
\begin{tabular}{lrl}
The Personal Robotics Lab & \textsc{Phone:} & (412) 973 9615 \\
The Robotics Institute& \textsc{Fax:} & (412) 297 4110 \\
Carnegie Mellon University& \textsc{Email:} & \url{siddh@cs.cmu.edu}\\
5000 Forbes Avenue&  \textsc{WWW:} & \url{http://www.cs.cmu.edu/~siddh}\\
Pittsburgh, PA - 15213 & \textsc{Admin:} & Keyla Cook (\url{keylac@cs.cmu.edu})
\end{tabular}
\end{center}


\section{Biosketch}
\noindent PI Srinivasa is the Boeing Endowed Professor at The Paul G. Allen School of Computer Science \& Engineering at the University of Washington, and an IEEE Fellow for contributions to robotic manipulation and human-robot interaction. He is a full-stack roboticist, with the goal of enabling robots to perform complex manipulation tasks under uncertainty and clutter, with and around people. To this end, he founded the Personal Robotics Lab in 2005. He is/was a PI on the Quality of Life Technologies NSF ERC, RCTA, DARPA ARM-S, DARPA RACER, and the DARPA Robotics Challenge, has built several robots (HERB, ADA, CHIMP, MuSHR), and has written software frameworks (OpenRAVE, DART) and best-paper award winning algorithms (CBiRRT, CHOMP, BIT*, Legibility, LazySP) used extensively by roboticists around the world. Srinivasa received a B.Tech in Mechanical Engineering from the Indian Institute of Technology Madras in 1999, and a PhD in 2005 from the Robotics Institute at Carnegie Mellon University.

% !TEX root = sidd-cv.tex

\section{Education}
\noindent
Ph.D., Carnegie Mellon University\hfill August 2005\\
Committee: Matthew Mason (advisor), Michael Erdmann (advisor),\\ 
Alfred Rizzi (CMU), Yan-bin Jia (Iowa State)\\
Thesis:  \textit{Control Synthesis for Dynamic Contact Manipulation}\\
\\
Master of Science, Carnegie Mellon University\hfill August 2001\\
Advisors: Matthew Mason, Michael Erdmann\\
Thesis:  \textit{Experiments with Nonholonomic Manipulation}\\
\\
Bachelor of Technology, Indian Institute of Technology Madras\hfill August 1999\\
Advisor: A. Radhakrishnan\\
Thesis: \textit{Reverse Engineering using the Structured Lighting Technique}

% !TEX root = sidd-cv.tex

\section{Employment}
\noindent
Boeing Endowed Professor in Computer Science \& Engineering \hfill 2017-\\
Computer Science \& Engineering Department, 
The University of Washington at Seattle\\
\\
Director of Robotics, Amazon Inc.\hfill 2018-\\
\\
First Wave Founder, Berkshire Grey Inc.\hfill 2014-18\\
\\
Finmeccanica Associate Professor in Computer Science\hfill 2013-17\\
Associate Professor, 
The Robotics Institute,  Carnegie Mellon University
\hfill 2011-13\\
\\
Senior Research Scientist, Intel Labs Pittsburgh \hfill 2005-11



\section{Honors and Awards}
\begin{itemize}
\addtolength{\itemsep}{-0.5\baselineskip}
\item ACM/HRI Best Demo Award Winner, 2024
\item ACM/HRI Best Paper Award Winner for Design~, 2023
\item ACM/HRI Best Paper Award Winner for Technical Advances in HRI, 2019
\item ICAPS Best Student Paper Award Winner, 2019
\item IEEE Fellow, 2018
\item ICAPS Best Paper Award Winner, 2018
\item ACM/IEEE HRI Best Paper Award Finalist, 2018
\item Boeing Endowed Professorship in Computer Science, 2017-23
\item CMU Women’s Association outstanding graduating senior advisor (Rachel Holladay), 2017
\item IEEE ICRA Best Vision Paper Award Finalist, 2016
\item RSS Best Systems Paper Award Finalist, 2015
\item IEEE ICRA Best Conference Paper Award Finalist, 2015
\item IEEE ICRA Best Video Award Finalist, 2014
\item Finmeccanica Chair in Computer Science, 2013-16
\item RSS Early Career Spotlight Award, 2013
\item RSS Best Paper Award Finalist, 2013
\item IEEE ICRA Best Manipulation Paper Award Finalist, 2013
\item Robotics Institute Cool Person of the Year Award, 2012
\item Okawa Foundation Research Grant, 2012
\item Office of Naval Research Young Investigator Award, 2012
\item IEEE RO-MAN Best Paper Award Finalist, 2012
\item RSS Best Paper Award Finalist, 2012
\item RAS Most Active Technical Committee Award: Mobile Manipulation, 2011 
\item ACM/IEEE HRI Best Paper Award Winner, 2010
\item IEEE IROS Best Paper Award Finalist, 2010
\item IEEE ICRA Best Manipulation Paper Award Finalist, 2010
\item IEEE ICRA Best Vision Paper Award Finalist, 2009
\item Intel Corporate Technology Group Divisional Recognition Awards, May, July, Oct 2008
\item Intel Research Pittsburgh Lab Research Awards, January, July 2006, January 2007
%\item Graduate Fellowship, The Robotics Institute, Carnegie Mellon University 1999 - 2005
%\item Jawaharlal Nehru Summer Research Fellowship, Indian Institute of Science 1997,1998
%\item Rajiv Gandhi Award for Best Summer Research Fellow, Indian Institute of Science 1997,1998
\item Indian National Mathematics Olympiad, 1994
\end{itemize}



\section{Five Most Relevant Publications}
\renewcommand{\labelenumi}{[ \arabic{enumi} ]\hfill}
\begin{enumerate}

\item \bibentry{kay2024ccil}

\item \bibentry{spencer2022eil}

\item \bibentry{nikolaidis2017mutual}

\item \bibentry{dragan2014legibility}

\item \bibentry{zucker2013chomp}


\end{enumerate}

\section{Five Other Significant Publications}
\renewcommand{\labelenumi}{[ \arabic{enumi} ]\hfill}
\begin{enumerate}

\item \bibentry{lim2024lgls}

\item \bibentry{koval2015pomdp}

\item \bibentry{collet2015herbdisc}

\item \bibentry{berenson2011task}

\item \bibentry{dragan2012shared}


\end{enumerate}

\section{Qualifications}
The PI has significant experience related to the proposed project including the following select projects:
\begin{itemize}
    \item \textbf{ONR SquadBot.} In this ongoing ONR-sponsored project, IHMC is developing the Nadia humanoid robot, and creating a “SquadBot” system, a humanoid robot system that demonstrates the potential of robots operating as part of a squad in urban operations, with the focus on building search. PI Srinivasa has contributes motion planning, grasping, and perception algorithms to the project. Work in this project will leverage the SquadBot platform as a potential demonstrator of the learning algorithms developed.
    \item \textbf{DARPA RACER (Aggressive, Resilient, High-speed Navigation in Off-road Terrain).} PI Srinivasa’s team at UW has been developing a framework for fluent motion planning for high-speed off-road navigation, integrating components such as fast replanning, contingency planning, and learning from experience.
    \item \textbf{DARPA Robotics Challenge (DRC).} PI Srinivasa was a lead PI of the the Carnegie Mellon UNiversity (CMU) team which built CHIMP, a four-limbed robot  which finished third in the overall challenge. PI Srinivasa was responsible for the motion planning, grasping, and manipulation stack of the robot.
    \item \textbf{DARPA Autonomous Robotic Manipulation - Software (ARM-S).} PI Srinivasa was a lead PI of the Carnegie Mellon University team which won the DARPA ARM-S challenge, a series of complex physical manipulation tasks including door opening, drilling, and changing the tyre of a car using an autonomous bimanual manipulator. PI Srinivasa's algorithms for the motion planning , manipulation, control, and shared autonomy were deployed on the robot.
    \item \textbf{Army Research Laboratory’s Robotics Collaborative Technology Alliance (RCTA).} PI Srinivasa’s UW team worked with NASA/JPL’s ROMAN platform, developing a motion planning stack for mobile manipulation. The team demonstrated the efficacy of their framework across a series of challenging real-world manipulation tasks including clearing debris on rough outdoor terrain. 
\end{itemize}

\newpage
\bibliographystyle{plain} 
% First uncomment the line below to generate the bbl, then comment it to remove Reference section
\bibliography{pubs/siddpubs-journal,pubs/siddpubs-conf,pubs/siddpubs-misc}
\end{document}
