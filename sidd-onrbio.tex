\documentclass[10pt]{article}

\RequirePackage{color,graphicx}
\usepackage{supertabular} 				%for Grades
\usepackage{titlesec}					%custom \section
\usepackage{palatino}
% Below sets font to Arial; acceptable fonts include 10pt Arial (space saver) or 11pt Times New Roman or Computer Modern 
%\renewcommand{\rmdefault}{phv} 
%\renewcommand{\sfdefault}{phv} 

\usepackage{bibentry}
\bibliographystyle{plain} 
\nobibliography*


\usepackage{tabularx}
%Setup hyperref package, and colours for links
\usepackage{hyperref}
\definecolor{linkcolour}{rgb}{0,0.2,0.6}
\hypersetup{colorlinks,breaklinks,urlcolor=linkcolour, linkcolor=linkcolour}
\usepackage{url}
\urlstyle{sf}

%CV Sections inspired by: 
%http://stefano.italians.nl/archives/26
\titleformat{\section}{\bf\large\raggedright}{}{0em}{}[\titlerule]
\titlespacing{\section}{0pt}{5pt}{5pt}
\titleformat{\subsection}{\bf\raggedright}{}{0em}{}
\titlespacing{\subsection}{0pt}{0pt}{10pt}
% Tweak page extents
\addtolength{\topmargin}{-8em}
\addtolength{\textheight}{8em}
\addtolength{\oddsidemargin}{-8em}
\addtolength{\evensidemargin}{-8em}
\addtolength{\textwidth}{16em}

% Using svn:keyword
% svn propset svn:keywords 'Date' cv-sidd-10.tex
\newcommand{\svn}[1]{\svnsub#1}
\def\svnsub$#1${#1}

%--------------------BEGIN DOCUMENT----------------------
\begin{document}

\pagestyle{empty} % non-numbered pages

%--------------------TITLE-------------
\par{\centering
		{\bf\LARGE Prof. Siddhartha Srinivasa
	}\bigskip\par}


%--------------------LAST MODIFIED-------------
%\let\thefootnote\relax\footnotetext{Last modified: \svn{$Date: 2012-12-09 21:46:45 -0500 (Sun, 09 Dec 2012) $}} 

%--------------------SECTIONS-----------------------------------
\begin{center}
\begin{tabular}{lrl}
The Personal Robotics Lab & \textsc{Phone:} & (412) 973 9615 \\
The Robotics Institute& \textsc{Fax:} & (412) 297 4110 \\
Carnegie Mellon University& \textsc{Email:} & \url{siddh@cs.cmu.edu}\\
5000 Forbes Avenue&  \textsc{WWW:} & \url{http://www.cs.cmu.edu/~siddh}\\
Pittsburgh, PA - 15213 & \textsc{Admin:} & Keyla Cook (\url{keylac@cs.cmu.edu})
\end{tabular}
\end{center}


\section{Education}
\noindent
Bachelor of Technology, Mechanical Engineering, Indian Institute of Technology Madras\hfill August 1999\\
Master of Science, Robotics, Carnegie Mellon University\hfill August 2001\\
Doctor of Philosophy, Robotics, Carnegie Mellon University\hfill August 2005\\

\section{Appointments}
\noindent
Boeing Endowed Professor, School of Computer Scinece \& Engineering, University of Washington \hfill 2017-Present\\
Finmeccanica Associate Professor, The Robotics Institute, Carnegie Mellon University \hfill 2011-2017\\
Senior Research Scientist, Intel Labs Pittsburgh \hfill 2005-2011\\

\section{Five Most Relevant Publications}
\renewcommand{\labelenumi}{[ \arabic{enumi} ]\hfill}
\begin{enumerate}

\item \bibentry{dragan2014legibility}

\item \bibentry{nikolaidis2017mutual}

\item \bibentry{strabala2013handoff}

\item \bibentry{zucker2013chomp}

\item \bibentry{dragan2012shared}


\end{enumerate}

\section{Five Other Significant Publications}
\renewcommand{\labelenumi}{[ \arabic{enumi} ]\hfill}
\begin{enumerate}

\item \bibentry{koval2015pomdp}

\item \bibentry{collet2015herbdisc}

\item \bibentry{paolini2014datadriven}

\item \bibentry{dragan2013teleop}

\item \bibentry{berenson2011task}


\end{enumerate}

\section{(d) Synergistic Activities}
\begin{itemize}
\addtolength{\itemsep}{-0.5\baselineskip}
\item \textbf{Curricula:} Developed two new graduate level courses, in robotic manipulation and robot autonomy: \\
\textbf{16-843 Manipulation Algorithms} Graduate-level course on the theory and algorithms that enable robots to physically manipulate their world including the geometry of manipulation configuration spaces, motion planning, synthesizing robust grasps for dexterous hands, reconfiguring clutter, and physics-based actions.\\
\textbf{16-662 Robot Autonomy}
Graduate-level course on manipulation, motion planning, perception, navigation,
and machine learning algorithms for mobile manipulators with strong hands-on component where students implement their assignments and class projects on a real mobile manipulation platform.
\item \textbf{Open-source Software, Hardware and Benchmarks:}
Barrett Technology ``puck'' motor controller; OWD, the Open WAM driver; COMPS, a constrained planning framework; MOPED, for object recognition and pose estimation; CHOMP, a gradient algorithm for trajectory optimization; GATMO, for navigation among movable objects; the YCB Object and Model Benchmark Dataset. The software is used by over $20$ research groups around the world. 
\item \textbf{Service.} \textbf{Chairs and Editorships:}
Editor, International Journal of Robotics Research 2013-;
Editor, IEEE/RSJ IROS 2014-16;
Program Chair RSS 2017,
Founding Chair, IEEE Robotics and Automation Soc. Technical Committee on Mobile Manipulation 2010-12;
Founding Program Chair, Robotics Track AAAI 2012-13;
Area Chair, RSS 2012-13;
Associate Editor, IEEE ICRA 2010-13;
Associate Editor, IEEE/RSJ IROS 2011-12.
\textbf{Workshops Organized:} 
IEEE ICRA 2015: Optimal Robot Motion Planning
IEEE ICRA 2015:Benchmarking in Manipulation Research: The YCB Object and Model Set,
IEEE IROS 2014: Rehabilitation and Assistive Robotics: Bridging the Gap Between Clinicians and Roboticists,
IEEE IROS 2014: Robot Manipulation: What has been achieved and what remains to be done?,
HRI 2013: Collaborative Manipulation: New Challenges for Robotics and HRI,
RSS 2012: Robots in Clutter: Manipulation, Perception and Navigation in Human Environments,
IEEE ICRA 2011:Mobile Manipulation: Integrating Perception and Manipulation, 
RSS 2010:Strategies and Evaluation for Mobile Manipulation in Household Environments,
ICCV 2009: Computer Vision for Humanoid Robots in Real Environments
and several others.
\textbf{Selected Program Committees:} 
Human Robot Interaction 2012, 13, 14, 15;
International Joint Conference on Artificial Intelligence (IJCAI) 2012;
International Conference on Automated Planning and Scheduling (ICAPS) 2010;
Robotics: Science and Systems (RSS) 2009, 10; 
AAAI Physically Grounded AI Track 2009, 11.
\item \textbf{Undergraduate Education:}
 The Personal Robotics Lab has engaged over 50 undergraduates at CMU, including students from underrepresented groups, through class projects, independent projects, workstudies during the school year, and REU summer research.  The lab developed a self-sustaining culture of attracting top undergraduate students in CS, ME, EECS and BME, and many of these students continue on to graduate school, some with NSF fellowships. 
 \textbf{Participation of underrepresented groups:}  Srinivasa is a mentor for the Robotics Summer Scholars Program for underrepresented students, and has mentored $10$ students since 2012 (4 female).
\item  \textbf{Public and K-12 outreach.} \textbf{Press:}
The Personal Robotics Lab generates significant interest from the general public, with features in National Geographic, Scientific American, Popular Science, Wired Magazine, PBS, and the BBC. HERB was named one of the ``World's most advanced robots'' in Businessweek. Srinivasa's work was featured on the nsf.gov website and on NSF Science Nation.    
 \textbf{Lab tours and talks:}
  HERB is a magnet for lab tours, with almost one tour a week. Srinivasa has hosted kindergarten, elementary, and high school groups, and given talks at local schools. The lab also demos at the annual National Robotics Week event for area high schools, which brings over 300 students for a day of lab tours and talks.
\end{itemize}

\newpage
\bibliographystyle{plain} 
% First uncomment the line below to generate the bbl, then comment it to remove Reference section
\bibliography{siddpubs/siddpubs-journal,siddpubs/siddpubs-conf,siddpubs/siddpubs-misc}
\end{document}
