\documentclass[10pt]{article}

\RequirePackage{color,graphicx}
\usepackage{supertabular} 				%for Grades
\usepackage{titlesec}					%custom \section
\usepackage{palatino}
% Below sets font to Arial; acceptable fonts include 10pt Arial (space saver) or 11pt Times New Roman or Computer Modern 
%\renewcommand{\rmdefault}{phv} 
%\renewcommand{\sfdefault}{phv} 

\usepackage{bibentry}
\bibliographystyle{plain} 
\nobibliography*


\usepackage{tabularx}
%Setup hyperref package, and colours for links
\usepackage{hyperref}
\definecolor{linkcolour}{rgb}{0,0.2,0.6}
\hypersetup{colorlinks,breaklinks,urlcolor=linkcolour, linkcolor=linkcolour}
\usepackage{url}
\urlstyle{sf}

%CV Sections inspired by: 
%http://stefano.italians.nl/archives/26
\titleformat{\section}{\bf\large\raggedright}{}{0em}{}[\titlerule]
\titlespacing{\section}{0pt}{5pt}{5pt}
\titleformat{\subsection}{\bf\raggedright}{}{0em}{}
\titlespacing{\subsection}{0pt}{0pt}{10pt}
% Tweak page extents
\addtolength{\topmargin}{-8em}
\addtolength{\textheight}{8em}
\addtolength{\oddsidemargin}{-8em}
\addtolength{\evensidemargin}{-8em}
\addtolength{\textwidth}{16em}

% Using svn:keyword
% svn propset svn:keywords 'Date' cv-sidd-10.tex
\newcommand{\svn}[1]{\svnsub#1}
\def\svnsub$#1${#1}

%--------------------BEGIN DOCUMENT----------------------
\begin{document}

\pagestyle{empty} % non-numbered pages

%--------------------TITLE-------------
\par{\centering
		{\bf\LARGE Prof. Siddhartha Srinivasa
	}\bigskip\par}


%--------------------LAST MODIFIED-------------
%\let\thefootnote\relax\footnotetext{Last modified: \svn{$Date: 2012-12-09 21:46:45 -0500 (Sun, 09 Dec 2012) $}} 

%--------------------SECTIONS-----------------------------------
\begin{center}
\begin{tabular}{lrl}
The Personal Robotics Lab & \textsc{Phone:} & (412) 973 9615 \\
The Robotics Institute& \textsc{Fax:} & (412) 297 4110 \\
Carnegie Mellon University& \textsc{Email:} & \url{siddh@cs.cmu.edu}\\
5000 Forbes Avenue&  \textsc{WWW:} & \url{http://www.cs.cmu.edu/~siddh}\\
Pittsburgh, PA - 15213 & \textsc{Admin:} & Keyla Cook (\url{keylac@cs.cmu.edu})
\end{tabular}
\end{center}


\section{Education}
\noindent
Ph.D., Carnegie Mellon University (CMU)\hfill August 2005\\
Advisors: Michael Erdmann \& Matthew Mason\hfill Thesis:  \textit{Control Synthesis for Dynamic Contact Manipulation}\\
B. Tech., Indian Institute of Technology Madras (IITM)\hfill August 1999\\
Advisor: A. Radhakrishnan\hfill
Thesis: \textit{Reverse Engineering using the Structured Lighting Technique}

\section{Appointments}
\noindent
Boeing Endowed Professor, School of Computer Scinece \& Engineering, University of Washington \hfill 2017-Present\\
Distinguished Engineer, Cruise \hfill 2022-Present\\
Director, Robotics AI, Amazon \hfill 2018-2022\\ 
Finmeccanica Associate Professor, The Robotics Institute, Carnegie Mellon University \hfill 2011-2017\\
Senior Research Scientist, Intel Labs Pittsburgh \hfill 2005-2011\\

\section{Biosketch}
\noindent Siddhartha Srinivasa is a Professor at The Paul G. Allen School of Computer Science \& Engineering at the University of Washington, and an IEEE Fellow. He is a full-stack roboticist, with the goal of enabling robots to perform complex manipulation tasks under uncertainty and clutter, with and around people. To this end, he founded the Personal Robotics Lab in 2005. He is a PI on the Quality of Life Technologies NSF ERC, RCTA, DARPA ARM-S, DARPA Robotics Challenge, and DARPA RACER, and has built several robots (HERB, ADA, CHIMP, MuSHR), and has written software frameworks (OpenRAVE, DART) and best-paper award winning algorithms (CBiRRT, CHOMP, BIT*, Legibility, LazySP) used extensively by roboticists around the world. Sidd received a B.Tech in Mechanical Engineering from the Indian Institute of Technology Madras in 1999, and a PhD in 2005 from the Robotics Institute at Carnegie Mellon University. He played badminton and tennis for IIT Madras, captained the CMU squash team, ran long-distance competitively, and lately plays tennis.

\section{Synergistic Activities}
\begin{itemize}
\addtolength{\itemsep}{-0.5\baselineskip}
\item \textbf{Open-source Software, Hardware and Benchmarks:}
Barrett Technology ``puck'' motor controller; OWD, the Open WAM driver; COMPS, a constrained planning framework; MOPED, for object recognition and pose estimation; CHOMP, a gradient algorithm for trajectory optimization; GATMO, for navigation among movable objects; the YCB Object and Model Benchmark Dataset. The software is used by over $20$ research groups around the world. 
\item \textbf{Service.} \textbf{Chairs and Editorships:}
Editor, International Journal of Robotics Research 2013-;
Editor, IEEE/RSJ IROS 2014-16;
Program Chair RSS 2017,
Founding Chair, IEEE Robotics and Automation Soc. Technical Committee on Mobile Manipulation 2010-12;
Founding Program Chair, Robotics Track AAAI 2012-13;
Area Chair, RSS 2012-13;
Associate Editor, IEEE ICRA 2010-13;
Associate Editor, IEEE/RSJ IROS 2011-12.
\item  \textbf{Public and K-12 outreach.} \textbf{Press:}
The Personal Robotics Lab generates significant interest from the general public, with features in National Geographic, Scientific American, Popular Science, Wired Magazine, PBS, and the BBC. HERB was named one of the ``World's most advanced robots'' in Businessweek. Srinivasa's work was featured on the nsf.gov website and on NSF Science Nation.    
 \textbf{Lab tours and talks:}
  HERB is a magnet for lab tours, with almost one tour a week. Srinivasa has hosted kindergarten, elementary, and high school groups, and given talks at local schools. The lab also demos at the annual National Robotics Week event for area high schools, which brings over 300 students for a day of lab tours and talks.
\end{itemize}

\newpage
\bibliographystyle{plain} 
% First uncomment the line below to generate the bbl, then comment it to remove Reference section
\bibliography{siddpubs/siddpubs-journal,siddpubs/siddpubs-conf,siddpubs/siddpubs-misc}
\end{document}
